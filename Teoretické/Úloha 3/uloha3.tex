\documentclass[a4paper, 12pt]{article}

\usepackage{amsmath}
\usepackage[total={17cm,25cm}, top=2.5cm, left=2cm, includefoot]{geometry}
\usepackage{enumitem}
\usepackage[T1]{fontenc}
\usepackage[bottom]{footmisc}

\setlist[itemize]{topsep=0pt}
\setlength{\itemindent}{0cm}
\setlength{\parskip}{9pt}
\setlength\parindent{0pt}

\begin{document}
  \section{P-I-3 Zahrádka}

  \subsection{Řešení pro $n\le100$}
  Pro každý bod lze projít všechny ostatní body, čímž vyčteme všechny možné dvojice bodů. Pro každou tuto dvojici lze opět vyčíst všechny ostatní body a pro každou takto vzniklou trojici vypočítat obsah trojúhelníku, který tvoří.

  Pro každý bod tedy uděláme následující:
  \begin{itemize}
    \item[a)] Posuneme souřadnice všech bodů tak, aby námi vybraný bod měl souřadnice $[0,0]$.
    \item[b)] Pro každou trojici bodů $[x_1,y_1]$, $[x_2,y_2]$ a $[0,0]$ spočítáme obsah jimi tvořeného trojúhelníku jako polovina délky jejich vektorového součinu.
  \end{itemize}

  Vektorový součin je však aplikovatelný pouze na body o třech souřadnicích, proto musíme k bodům přirat třetí souřadnici s nulovou hodnotou:

  $$
  S=\frac{| [x_1,y_1,0]\times[x_2,y_2,0] |}{2}
  =\frac{| [y_1 \cdot 0 - 0 \cdot y_2, 0 \cdot x_2 - x_1 \cdot 0, x_1 \cdot y_2 - y_1 \cdot x_2] |}{2}
  =\frac{| x_1 \cdot y_2 - y_1 \cdot x_2 |}{2}
  $$

  Nejmenší z takto vytvořených trojúhelníků je řešením našeho problému. Časová složitost výše popsaného algoritmu je $O(n^3)$, protože procházíme všechny trojice z dané množiny bodů.

  \subsection{Řešení pro $n\le1000$}
  Problém s algoritmem navrhovaným pro $n\le100$ je ten, že pro každý bod kontrolujeme všechny dvojice tvořeny ostatními body.

  Z výrazu $\frac{| x_1 \cdot y_2 - y_1 \cdot x_2 |}{2}$ je patrné, že k nalezení trojúhelníku s nejmenším obsahem potřebujeme minimalizovat výraz v absolutní hodnotě.

  Proto místo toho, abychom zkoušeli všechny dvojice, můžeme zkusit pro každý bod setřídit všechny ostatní body porovnáváním $x_1 \cdot y_2 < y_1 \cdot x_2$ (nebo $>$). Poté stačí vypočítat obsahy trojúhelníků tvořenými dvojicemi body, které spolu v setřízeném poli sousedí a bodem $[0,0]$.

  Pro každý bod tedy uděláme následující:
  \begin{itemize}
    \item[a)] Posuneme souřadnice všech bodů tak, aby námi vybraný bod měl souřadnice $[0,0]$.
    \item[b)] Všechny body (kromě bodu $[0,0]$) setřídíme porovnáváním $x_1 \cdot y_2 < y_1 \cdot x_2$ a po setřízení pro každé dva sousední body a bod $[0,0]$ spočítáme obsah jimi tvořeného trojúhelníku (opět pomocí poloviny vektorového součinu).
  \end{itemize}

  Nejmenší z takto vytvořených trojúhelníků je řešením našeho problému.

  Je nutno dodat, že je možné, že pro daný bod zvolený v části $(a)$ našeho algoritmu nebude nejmenší jím (a dalšími dvěma body) tvořený trojúhelník nalezen. Tento trojúhelník bude ovšem nalezen při zvolení jednoho z dalších dvou bodů, které nenalezený trojúhelník tvoří.

  Časová složitost výše popsaného algoritmu je $n^2 \cdot log(n)$, protože pro každý z množiny bodů třídíme (v $O(n \cdot log(n))$) všechny ostatní body.
\end{document}
