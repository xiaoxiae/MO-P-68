\documentclass[a4paper, 12pt]{article}

\usepackage{amsmath}  % Math
\usepackage[total={17cm,25cm}, top=2.5cm, left=2cm, includefoot]{geometry}
\usepackage{enumitem} % Lists
\usepackage[T1]{fontenc}  % Formating czech symbols

\setlist[itemize]{topsep=0pt}
\setlength{\itemindent}{0cm}
\setlength{\parskip}{9pt}
\setlength\parindent{0pt}

\begin{document}
  \section{P-I-4 Dva lupiči}
  Nechť jsou pro řešení zadání $(a)$, $(b)$ a $(c)$ dány následující pojmy:
  \begin{itemize}[noitemsep]
    \item Vstup jako posloupnost různých hodnot věcí $v=p_1, p_2,...p_n;n \in N, p \in R^+$.
    \item Součet hodnot věcí $s=\sum\limits_{i=1}^n p_i$ pro daný vstup o velikosti $n$.
    \item Vstup $v_{opt}$ pro který platí, že OPT$(v_{opt})=s/2$ (tj. vstup produkující optimální výstup\footnote{\label{note1}V rámci našeho problému musí být optimální výstup $s/2$, protože lépe než na dvě stejné poloviny žádnou skupinu věcí rozdělit nejde.}).
  \end{itemize}

  \subsection{Řešení zadání $(a)$}
  Pro zadání $(a)$ dává ALG pro libovolný vstup $v$ všechen lup na první hromadu. Proto se největší hromada musí při každém vstupu rovnat součtu hodnot věcí $s$, protože na druhé hromadě nic není.

  Pro optimální algoritmus s optimálním vstupem platí, že OPT$(v_{opt})=s/2$.
  ALG$(v)=s$ pro libovolný vstup, takže musí rovněž platit, že ALG$(v_{opt})=s$. Úpravou získáme rovnost ALG$(v_{opt}) = 2 \cdot$OPT$(v_{opt})$.

  OPT$(v_{opt})$ vždy produkuje optimální výstup\footnotemark[\ref{note1}], takže platí nerovnost OPT$(v_{opt}) \le$ OPT$(v)$. Úpravou nerovnosti získáváme:

  \begin{center}
    $2 \cdot$OPT$(v_{opt}) \le 2\cdot$OPT$(v)$

    ALG$(v) \le 2 \cdot$OPT$(v)$
  \end{center}

  Z poslední úpravy je patrné, že výše popsaný algoritmus je 2-kompetitivní.

  \subsection{Řešení zadání $(b)$}
  Mějme algoritmus ALG$(v)$, který věci dává na hromádky následovně:
  \begin{itemize}[noitemsep, topsep=0pt]
    \item Pokud hromady nejsou stejné, umístí prvek na menší hromádku.
    \item Pokud hromádky stejné jsou, umístí prvek na první hromádku.
  \end{itemize}

  Pojďme si uvedený algoritmus rozebrat pro vstupy $v$ o různých délkách vstupu $n$.

  \subsubsection{Řešení pro $n \le 2$}
  Pokud platí, že $n \le 2$, tak se pro posloupnost $v=p_1,...,p_n$ algoritmus ALG$(v)$ chová stejně jako optimální algoritmus OPT$(v)$:
  \newpage
  \begin{equation*}
      \text{OPT}(v)=\text{ALG}(v)= \begin{cases}
        0 & n=0
        \\
        p_1 & n=1
        \\
        \text{MAX\footnotemark}(p_1, p_2) & n=2
      \end{cases}
  \end{equation*}
  \footnotetext{Funkce MAX$(a, b)$ vrací větší ze dvou vstupních čísel.}

  ALG první dva prvky umístí na dvě rozdílné hromádky. OPT se vždy chová stejně nebo lépe než ALG, takže mu v tomto případě nezbývá nic jiného, než dělat to samé.

  \subsubsection{Řešení pro $n \ge 3$}

  \subsection{Řešení zadání $(c)$}
  Uvažujme libovolný algoritmus ALG, pro který zkonstruujeme optimální vstup $v_{opt}$ následujícím způsobem:
  \begin{center}
    $p_1=1, p_2=1+a,..., p_n=\sum\limits_{i=1}^{n-1} p_i; n \in N, a \in R^+$
  \end{center}

  Prvních několik prvků vstupu je tedy $1, 1+a, 2+a, 4+2a, 8+4a, 16+8a, 32+16a$. Pro tento vstup platí, že OPT$(v_{opt})=s/2$, protože díky znalosti celého vstupu optimální algoritmus ví, že součet prvních $n-1$ prvků je roven $n$-tému prvku.

  Uvažovaný ALG však vstup dopředu nevidí, proto vždy musí umísťovat prvky do té menší z hromádek. Pokud by tak neučinil, zkonstruovali bychom stejným způsobem nový vstup o takové délce $n$, aby se ALG zastavil v tom bodě, ve kterém prvek do menší z hromádek neumístil. Výsledek pro tento vstup by pro poměr s optimálním algoritmem OPT pro libovolné $a$ přesahoval námi hledanou hranici $3/2$.

  Umísťuje-li algoritmus prvky na tu menší ze dvou hromádek, tak pro náš vstup musí produkovat výstupy odpovídající poslupnosti $1, 1+a, 3+a, 5+3a...$. Pokud je dáme do poměru se součtem hodnot $s$, dostáváme:

  \begin{center}
    $\frac{1}{1}, \frac{1+a}{2+a}, \frac{3+a}{4+2a}, \frac{5+3a}{8+5a}, \frac{11+5a}{16+8a}...$
  \end{center}

  Přiblížíme-li $a$ libovolně blízko nule, výsledky algoritmu ALG v poměru s $s$ pro $n \ge 3$ budou:

  \begin{center}
    $\frac{1}{1}s, \frac{1}{2}s, \frac{3}{4}s, \frac{5}{8}s, \frac{11}{16}s...$.
  \end{center}

  Z výsledku $3/4s$ je patrné, že libovolný ALG pro vstup námi zkonstruovaný vstup $v_{opt}$ o velikosti $n=3$ musí vyprodukovat výsledek, který bude s optimálním řešením v poměru:
  \begin{center}
    $\frac{ALG(v_{opt})}{OPT(v_{opt})}=\frac{3/4s}{1/2s}=3/2$
  \end{center}

  Proto pro libovolný on-line algoritmus ALG na dělení lupu proto platí, že:
  \begin{center}
    ALG$(v) \le \frac{3}{2}$OPT$(v)$
  \end{center}

\end{document}
