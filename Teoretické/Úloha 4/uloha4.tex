\documentclass[a4paper, 12pt]{article}

\usepackage{amsmath}
\usepackage[total={17cm,25cm}, top=2.5cm, left=2cm, includefoot]{geometry}
\usepackage{enumitem}
\usepackage[T1]{fontenc}
\usepackage[bottom]{footmisc}

\setlist[itemize]{topsep=0pt}
\setlength{\itemindent}{0cm}
\setlength{\parskip}{9pt}
\setlength\parindent{0pt}

\title{P-I-4 Dva lupiči}
\author{Tomáš Sláma}
\date{}

\begin{document}
  \maketitle

  \bigskip
  \begin{center}
    \textsc{Vrát 19, Železný Brod, 468 22}

    \textsc{Počet stran řešení: 2}
  \end{center}

  \pagenumbering{gobble}
  \newpage
  \pagenumbering{arabic}

  \section{P-I-4 Dva lupiči}
  Nechť jsou pro řešení zadání $(a)$, $(b)$ a $(c)$ dány:
  \begin{itemize}[noitemsep]
    \item Vstup $v$ jako posloupnost různých hodnot věcí $v=p_1, p_2,...p_n; n \in N$.
    \item Součet hodnot věcí $s=\sum\limits_{i=1}^n p_i$ pro daný vstup o velikosti $n$.
    \item Vstup $v_{opt}$ pro který platí, že OPT$(v_{opt})=s/2$ (tj. vstup produkující optimální výstup\footnote{V rámci našeho problému musí být optimální výstup $s/2$ (lépe než na dvě poloviny věci rozdělit nejde).}).
    \item Polovina rozdílu velikostí hromádek $r$.
  \end{itemize}

  \subsection{Řešení zadání $(a)$}
  Pro zadání $(a)$ dává ALG pro libovolný vstup $v$ všechen lup na první hromadu. Proto se hromada s větší hodnotou musí při každém vstupu rovnat součtu hodnot věcí $s$ (na druhé hromadě nic není).

  Pro optimální algoritmus s optimálním vstupem platí, že OPT$(v_{opt})=s/2$.
  ALG$(v)=s$ pro libovolný vstup, takže musí rovněž platit, že ALG$(v_{opt})=s$. Úpravou získáme rovnost ALG$(v_{opt}) = 2 \cdot$OPT$(v_{opt})$.

  OPT$(v_{opt})$ vždy produkuje optimální výstup, takže platí nerovnost $s/2 \le$ OPT$(v)$. Úpravou nerovnosti získáváme:

  $$s \le 2\cdot\text{OPT}(v)$$
  $$\text{ALG}(v) \le 2 \cdot\text{OPT}(v)$$

  Z poslední úpravy je patrné, že výše popsaný algoritmus je 2-kompetitivní.

  \subsection{Řešení zadání $(b)$}
  Mějme algoritmus ALG$(v)$, který věci dává na hromádky následovně:
  \begin{itemize}
    \item Pokud hromady nejsou stejné, umístí prvek na hromádku s menší hodnotou.
    \item Pokud hromádky stejné jsou, umístí prvek na libovolnou z obou hromádek.
  \end{itemize}

  \subsubsection{Důkaz 3/2 kompetitivnosti}
  Pro výstup algoritmu OPT pro libovolný vstup $v$ platí:
  \begin{itemize}
    \item OPT$(v) \ge s/2$ (viz. úloha $(a)$).
    \item OPT$(v) \ge p \; \forall \; p \in v$ (je zřejmé, že OPT nemůže nikdy vyprodukovat nic menšího než hodnotu největšího prvku vstupu).
  \end{itemize}

  Pro výstup algoritmu ALG pro libovolný vstup $v$ platí:
  \begin{itemize}
    \item Pokud se před přidáním posledního prvku hromádky rovnají, tak pro poslední prvek posloupnosti musí platit, že $p_n=2r$ (je dvojnásobek poloviny rozdílu hromádek). Pokud se nerovnají, tak je $p_n$ nutně větší než rozdíl hromádek: $p_n \ge 2r$.
    \item ALG$(v)=r+\frac{s}{2}$ (vždy produkuje polovinu celkového součtu plus polovinu rozdílu hromádek).
  \end{itemize}

  OPT$(v) \ge p \; \forall \; p \in v$, což musí platit i pro $p_n$. Dosazením do nerovnosti dostáváme:

  $$\text{ALG}(v)=r+\frac{s}{2}$$
  $$\text{ALG}(v) \le \frac{\text{OPT}(v)}{2}+\text{OPT}(v)$$
  $$\text{ALG}(v) \le \frac{3}{2}\text{OPT}(v)$$

  Kompetitivnost výše popsaného algoritmu je tedy $3/2$.

  \subsection{Řešení zadání $(c)$}
  Uvažujme vstup $v_{opt}=1,1+a,2+a; a \in R^+$ (je $opt$, protože součet prvních dvou se rovná tomu třetímu), jeho součet $s=4+2a; a \in R^+$ a libovolný on-line algoritmus ALG.

  Pro optimální algoritmus s námi zkonstruovaným vstupem platí, že OPT$(v_{opt})=s/2$ (vidí celý vstup).

  ALG však dopředu nevidí, že vstup takto vypadá, proto musí (mimo první prvek, který dá na libovolnou hromádku) umisťovat prvky vždy na hromádku s menší hodnotou. Pokud by dva po sobě jdoucí prvky umístil na stejnou hromádku, vstup bychom zkrátili tak, aby se ALG zastavil právě v tomto bodě.

  Pro náš vstup by se to mohlo stát mezi 1. a 2. prvky, kdy by tedy nejmenší ALG v poměru s OPT vyšel $\frac{2+a}{1+a}$ čímž bychom pro libovolné $a$ přesáhli hranici $3/2$. Další případ by mohl nastat mezi prvky č. 2 a 3, pro který by ALG v poměru s OPT vyšel $\frac{3+2a}{2+a}$, což by se pro $a$ blížící se nule přibližovalo ke $3/2$.

  Pokud námi uvažovaný vstup bude dávat prvky vždy na tu menší hromádku, musí platit, že ALG$(v_{opt})=3+a$ (1. a 3. prvek). Pro $a$ přibližující-se k nule vychází (v poměru s výsledkem optimálního algoritmu OPT) hodnoty:

  $$\lim_{a\to0} \frac{\text{ALG}(v_{opt})}{\text{OPT}(v_{opt})}
  =\lim_{a\to0} \frac{3+a}{2+a}
  =\frac{3}{2}$$

  Z tohoto výsledku je patrné, že každý on-line algoritmus lze pro vstup hotoho formátu "donutit" do výsledku, který je v poměru s optimálním výsledkem roven hodnotě $3/2$.
\end{document}
