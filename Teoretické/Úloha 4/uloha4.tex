\documentclass[a4paper, 12pt]{article}

\usepackage{amsmath}
\usepackage[total={17cm,25cm}, top=2.5cm, left=2cm, includefoot]{geometry}
\usepackage{enumitem}
\usepackage[T1]{fontenc}
\usepackage[bottom]{footmisc}

\setlist[itemize]{topsep=0pt}
\setlength{\itemindent}{0cm}
\setlength{\parskip}{9pt}
\setlength\parindent{0pt}

\begin{document}
  \section{P-I-4 Dva lupiči}
  Nechť jsou pro řešení zadání $(a)$, $(b)$ a $(c)$ dány následující pojmy:
  \begin{itemize}[noitemsep]
    \item Vstup $v$ jako posloupnost různých hodnot věcí $v=p_1, p_2,...p_n;n \in N, p \in R^+$.
    \item Součet hodnot věcí $s=\sum\limits_{i=1}^n p_i$ pro daný vstup o velikosti $n$.
    \item Vstup $v_{opt}$ pro který platí, že OPT$(v_{opt})=s/2$ (tj. vstup produkující optimální výstup\footnote{V rámci našeho problému musí být optimální výstup $s/2$.}).
  \end{itemize}

  \subsection{Řešení zadání $(a)$}
  Pro zadání $(a)$ dává ALG pro libovolný vstup $v$ všechen lup na první hromadu. Proto se hromada s větší hodnotou musí při každém vstupu rovnat součtu hodnot věcí $s$ (na druhé hromadě nic není).

  Pro optimální algoritmus s optimálním vstupem platí, že OPT$(v_{opt})=s/2$.
  ALG$(v)=s$ pro libovolný vstup, takže musí rovněž platit, že ALG$(v_{opt})=s$. Úpravou získáme rovnost ALG$(v_{opt}) = 2 \cdot$OPT$(v_{opt})$.

  OPT$(v_{opt})$ vždy produkuje optimální výstup, takže platí nerovnost OPT$(v_{opt}) \le$ OPT$(v)$. Úpravou nerovnosti získáváme:

  $$2 \cdot\text{OPT}(v_{opt}) \le 2\cdot\text{OPT}(v)$$
  $$\text{ALG}(v) \le 2 \cdot\text{OPT}(v)$$

  Z poslední úpravy je patrné, že výše popsaný algoritmus je 2-kompetitivní.

  \subsection{Řešení zadání $(b)$}
  Mějme algoritmus ALG$(v)$, který věci dává na hromádky následovně:
  \begin{itemize}[noitemsep, topsep=0pt]
    \item Pokud hromady nejsou stejné, umístí prvek na hromádku s menší hodnotou.
    \item Pokud hromádky stejné jsou, umístí prvek na libovolnou z obou hromádek.
  \end{itemize}

  \iffalse
  Pojďme si uvedený algoritmus rozebrat pro vstupy $v$ o různých délkách vstupu $n$.

  \subsubsection{Řešení pro $n \le 2$}
  Pokud platí, že $n \le 2$, tak se pro posloupnost $v=p_1,...,p_n$ algoritmus ALG$(v)$ chová stejně jako optimální algoritmus OPT$(v)$:
  \newpage
  \begin{equation*}
      \text{OPT}(v)=\text{ALG}(v)= \begin{cases}
        0 & n=0
        \\
        p_1 & n=1
        \\
        \text{MAX\footnotemark}(p_1, p_2) & n=2
      \end{cases}
  \end{equation*}
  \footnotetext{Funkce MAX$(a, b)$ vrací větší ze dvou vstupních čísel.}

  ALG první dva prvky umístí na dvě rozdílné hromádky. OPT se vždy chová stejně nebo lépe než ALG, takže mu v tomto případě nezbývá nic jiného, než dělat to samé.
  \fi

  \subsection{Řešení zadání $(c)$}
  Uvažujme vstup $v_{opt}=1,1+a,2+a; a \in R^+$ (je $opt$, protože součet prvních dvou se rovná tomu třetímu), jeho součet $s=4+2a; a \in R^+$ a libovolný on-line algoritmus ALG.

  Pro optimální algoritmus platí, že OPT$(v_{opt})=s/2$ (vidí, že si musí nechat místo na třetí prvek v jedné z hromádek).

  ALG však dopředu nevidí, že vstup takto vypadá, proto musí (mimo první prvek, který dá na libovolnou hromádku) umisťovat prvky vždy na hromádku s menší hodnotou. Pokud by dva po sobě jdoucí prvky umístil na stejnou hromádku, vstup bychom zkrátili tak, aby se ALG zastavil právě v tomto bodě.

  Pro náš vstup by se to mohlo stát mezi 1. a 2. prvky, kdy by tedy nejmenší ALG v poměru s OPT vyšel $\frac{2+a}{1+a}$ čímž bychom přesáhli hranici $3/2$. Další případ by mohl nastat mezi prvky č. 2 a 3,

  Pro námi uvažovaný vstup tedy musí platit $ALG(v_{opt})=3+a$ (1. a 3. prvek). Pro $a$ přibližující-se k nule vychází (v poměru s výsledkem optimálního algoritmu OPT) hodnoty:

  $$\lim_{a\to0} \frac{\text{ALG}(v_{opt})}{\text{OPT}(v_{opt})}
  =\lim_{a\to0} \frac{3+a}{2+a}
  =\frac{3}{2}$$

  Z tohoto výsledku je patrné, že každý on-line algoritmus lze "donutit" do výsledku, který je v poměru s optimálním výsledkem roven hodnotě $3/2$.

  Proto platí, že: $$\text{ALG}(v) \le \frac{3}{2}\text{OPT}(v)$$
\end{document}
